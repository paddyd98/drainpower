\section{Executive Summary}
DrainPower is an innovative company that aims to transform energy efficiency in residential heating by harnessing and recycling the thermal energy that is usually lost in sewage systems. The Drain Water Heat Pump (DWHP) is a cutting-edge product that stands out from conventional heating systems and other heat pump technologies by offering exceptional efficiency of up to 450\%, even in colder climates.

The power of DrainPower resides not solely in our state-of-the-art technology, but also in our varied and exceptionally proficient team. The founding team comprises individuals from six distinct nationalities, thus fostering a diverse amalgamation of cultural viewpoints and engineering proficiency. The presence of diverse perspectives and backgrounds encourages the generation of new ideas and original approaches, allowing us to tackle problems from various perspectives and create solutions that are genuinely distinctive in the market.

Our team comprises individuals with diverse engineering backgrounds, such as mechanical, electrical, and environmental engineering. This diversity enables us to possess the necessary technical expertise to design, develop, and enhance our products. Our extensive network of industry connections, especially in the hotel sector, complements our robust technical foundation. Our co-founder, Patrick Doran, has extensive experience and connections in the hotel industry, including established collaborations and potential partnerships with prominent companies such as Tifco Hotels and the Marriott Group. These connections not only furnish us with valuable market insights, but also substantially diminish our marketing expenses and expedite market penetration.

DrainPower is launching into the market during a crucial period when there is a heightened urgency for energy efficiency and sustainability. Due to the escalating energy expenses and mounting regulatory constraints on carbon emissions, there is an expanding need for inventive solutions capable of addressing these issues. Our product directly tackles this need, providing substantial cost reductions and environmental advantages to both commercial and residential customers.

\textbf{In summary}, DrainPower combines a game-changing technology with a skilled and diverse team, extensive industry relationships, and a product that effectively tackles a significant global problem. Our strengths enable us to not only compete but also take the lead in the rapidly expanding market for energy-efficient heating solutions. We have full assurance that DrainPower will have a significant influence on the management of energy in both households and commercial buildings, establishing a fresh benchmark for sustainability in the industry.


\section{Company Description}

DrainPower is a company that focuses on improving energy efficiency in households by using the thermal energy that is usually wasted in sewage systems. Our main product is a drain water heat pump that provides a new way to heat water more efficiently and sustainably.

Normally, when households use water, the leftover warm water, which is around 20°C, just goes down the drain and is lost in the sewage system. Our solution, as seen in Figure \ref{fig:DWHP}, captures this heat before it is wasted. The system then uses electricity to compress this captured energy, making it stronger. After that, the energy is put back into the household water heating system, which makes it much more efficient. In fact, our system can reach about 450\% efficiency.

Our system works in three main stages. First, it captures the thermal energy from the drain water. Next, the energy is compressed using electricity to increase its heating potential. Finally, this compressed energy is used to reheat the household water, effectively reusing energy that would otherwise be lost.

In addition to this, our system includes temperature sensors installed on the drain water system. These sensors allow us to incorporate an AI optimization platform that further reduces energy waste and usage. The AI learns from past consumption patterns and predicts future usage, enabling the system to operate even more efficiently by adapting to the household’s specific needs.

DrainPower’s technology represents a significant advancement in how energy can be used more sustainably in everyday life. By preventing thermal energy from going to waste, our system helps households save energy and reduce costs, while also being environmentally friendly. We believe that our innovative approach can make a substantial difference in how energy is managed in homes.

\begin{figure}[ht]
    \centering
    \includegraphics[width=0.75\linewidth]{Pictures/DWHP_Function.png}
    \caption{Schematic of the Drain Water Heat Pump}
    \label{fig:DWHP}
\end{figure}

\section{Opportunity}

In colder regions and larger buildings with high occupancy rates, traditional outdoor heating units often fail to maintain optimal energy efficiency. This inefficiency leads to significantly higher heating costs, exacerbating both economic and environmental challenges. According to the European Commission, space heating accounts for approximately 64\% of energy consumption in residential buildings across the European Union, with this percentage rising in colder climates \cite{eurostat_energy_nodate}.

In these regions, the demand for heating is not only greater but also more difficult to manage, especially in larger buildings where the energy needs are substantial. The inefficiency of traditional systems results in increased operational expenses, which can be a considerable burden for building owners and occupants alike. Furthermore, the reliance on less efficient heating methods contributes to higher greenhouse gas emissions. The EU has identified the residential and commercial sectors as key areas for reducing carbon emissions, as they are responsible for about 36\% of CO2 emissions in Europe \cite{european_environment_agency_greenhouse_nodate}.

This situation underscores the urgent need for more sustainable heating solutions that can effectively meet the demands of larger spaces while minimising both costs and environmental impact. By improving energy efficiency and integrating advanced technologies, such as heat recovery and AI-driven optimisation, we can address these challenges and support the EU’s goals of reducing energy consumption and greenhouse gas emissions \cite{european_commission_2050_nodate}.

The ideal customer for our project includes owners or managers of larger buildings such as commercial complexes, educational institutions, hotels, or recreational facilities located in colder regions. At the same time, house owners, especially those residing in colder regions, can also benefit. While larger buildings with high occupancy are a primary target due to the significant heating demand, homeowners in colder climates face similar challenges with traditional heating systems. They are individuals or entities concerned about rising heating costs and the environmental impact of their energy consumption. Their properties experience high occupancy rates, leading to a significant demand for efficient heating solutions that can accommodate these larger spaces. These customers are seeking innovative ways to decrease their energy bills while aligning with sustainable practices to reduce their ecological footprint. They are open to exploring new technologies that enhance energy efficiency and would consider investing in long-term solutions that offer both financial savings and environmental benefits. 

Commercial property owners, especially those managing larger buildings with numerous occupants, face similar challenges on a larger scale, striving to provide adequate heating while keeping costs and environmental impact in check. Moreover, for residential homeowners, the high energy bills during colder months due to inefficient heating systems are a concern. Building heating is a need that all the interviewed customers have, both private and commercial. However, the level of importance varies a little depending on the location of the interviewed people. As an example, for Mr. Doran, that operates his business in the Black Forest in Germany where it gets very cold, has a high need for an efficient and reliable heating system [Interview Mr. Doran] 

  

Besides the basic need for a heating system, that can be satisfied by a range of different technologies, there is also a strive to decrease the economic costs of energy consumption [Interview Mr. Faasen, Mr. Doran], as well as their environmental footprint [Interview Mr. Doran].  

 

Problems that arise when striving towards this are: 

\begin{enumerate}
    \item Limited time to research and work out further implementation steps.
    \item No further instalment of heat pump allowed due to a full energy grid.
    \item Too high investment costs to gain too small efficiencies, payback time is too large for new technologies.
    \item Old current energy systems in place, with limited flexibility.
\end{enumerate}

 

Besides the validation through the interviews, the 2023 market report of the European-Heat-Pump-Association (EHPA) that was provided to us by their Head of Communications gave us an interesting insight into the layout of the general heat pump market \cite{european_heat_pump_association_2023_nodate}. As an indicator, the sales on the heat pump market have quadrupled in the past 10 years. Furthermore, air-to-air heat pumps currently account for approximately 80\% of the market volume. A reason for this would be the ease of installation and the lower investment cost. Therefore, these are also factors to be considered in the customer sector validation.


\section{Business Model}
   

The provided Business Model Canvas, Figure \ref{fig:BMC}, shows a comprehensive outline of the company’s strategy.  

\begin{figure}[ht]
    \centering
    \includegraphics[width=1\linewidth]{Pictures/BMC.png}
    \caption{Business Model Canvas}
    \label{fig:BMC}
\end{figure}

Key partners include manufacturers, assemblers and heat pump manufacturers, demonstrating a robust supply chain to produce efficient heating solutions. The company's core activities focus on the design and development of the DWHP evaporator through proactive (PUSH) strategies, while also engaging in marketing and promotional efforts (PULL) to increase market penetration. Another activity is maintenance training, which aims to ensure long-term customer satisfaction and system efficiency, as well as a continuous income stream.  

The key resources are rooted in the company's technological and intellectual property related to DWHP, supported by a research and development team as well as marketing and sales teams. The value proposition highlights several competitive advantages, including increased energy independence, heat pump efficiency improvements of around 35\% and consistent operation at low temperatures. The system also promises reduced energy costs for end users, just-in-time delivery of optimisation services and a comprehensive optimisation platform.  


The customer relationship strategy, like mentioned above, includes maintenance services, online support, and an optimisation platform, ensuring continuous engagement and satisfaction. There are different channels for reaching customers, including direct sales (B2B), collaborations with hotel chains, participation in energy and hotel fairs, and social media outreach, specifically via LinkedIn. Target customer segments are identified as hotel owners and hotel chains, which can significantly benefit from the energy efficiency and cost savings provided by the DWHP system. 

 

The cost structure consists of R\&D, manufacturing and production costs, patent costs, marketing, and promotion costs and maintenance training. The revenue stream is generated from the sale of heat pumps, as well as the continuous income stream consisting of subscription fees for maintenance services and the optimisation platform.  

External factors include regulatory approvals, competition from other energy-efficient solutions and economic influences on the construction and renovation markets.  

 

Below shows a simplified outline of the company’s ecosystem, to clarify the operation of DrainPower.

\begin{figure}[ht]
    \centering
    \includegraphics[width=0.75\linewidth]{Pictures/Ecosystem.png}
    \caption{DrainPower's Ecosystem}
    \label{fig:enter-label}
\end{figure}

\section{Business Model Implementation}

\subsection{Marketing Plan}
DrainPower’s marketing plan is designed to establish a strong presence in the European heat pump market by targeting key commercial sectors, particularly the hotel industry. The plan leverages the company’s competitive advantages and existing networks to effectively position DrainPower’s innovative heat pump solutions as the preferred choice for energy-efficient heating.

\subsubsection{Market Positioning}
DrainPower is positioned in the European heating market, specifically within the heat pump sector. We have analysed the main competitors in this market, which are shown in Figure \ref{fig:Competitor} below.

In our comparison, we look at traditional heating systems and air-source heat pumps. Traditional systems tend to maintain steady efficiency, but they are limited to around 90\%. On the other hand, air source heat pumps can reach up to 350\% efficiency, but only in perfect conditions. Their efficiency drops significantly in colder climates. Our system, however, provides a stable efficiency of about 450\%, even in colder weather.

There are other products available, like HuberTech's RoWin. But these systems are much bigger and cannot be easily installed in existing buildings without a lot of effort. Because of this, DrainPower fills an important gap in the market by offering a solution that is both efficient and easy to integrate into existing structures.


\begin{figure}[ht]
    \centering
    \includegraphics[width=0.75\linewidth]{Pictures/Competitor.png}
    \caption{Competitor Analysis}
    \label{fig:Competitor}
\end{figure}

The size of this market was analysed via the Total Addressable Market, Serviceable Addressable Market, and Serviceable Obtainable Market framework, finding the following market sizes: 

\begin{itemize}
\item TAM: €15.8B - European Heat Pump Market Size Valuation \cite{global_market_insights_europe_nodate}.
    \item SAM: €1.2B Narrowed down to colder regions and commercial buildings (hospitals, hotels, universities) that offer a sufficient drain water stream \cite{plos_climate_population_nodate} \cite{europen_comission_europes_nodate}.
    \item SOM: €100M Narrowed down to a more realistic factor of actual clients obtainable.
\end{itemize}


With the market target set (larger commercial buildings), the marketing strategy is to emphasise DrainPowers higher efficiency (adding value of lower heating costs and higher environmental savings), persuading customers to opt for DrainPower as if there is actually no other option.  

\subsubsection{Marketing Channels}

The channels used for this are all B2B, with DrainPower focussing on connecting with larger commercial buildings, particularly in the hotel industry. The strategy involves starting with discounted commercial MVP testing through collaborations with individual hotels. The goal is to establish connections and secure deals with larger hotel chains, which will streamline the implementation phase and create a strong network to roll out DrainPower’s Heat Pump. The initial outreach to hotels will be conducted through the following methods:

\begin{enumerate}
\item Participation in energy and hotel fairs.
\item Direct contact through our founders’ extensive network of hotel owners.
\item LinkedIn presence to generate referrals, share promotional success stories, and conduct cold calls.
\end{enumerate}

More specifically, Patrick Doran, a founder of DrainPower, comes from a family deeply rooted in the hotel industry. He has already begun working on collaborations with his family-owned hotel in Germany and has established connections with Enda O'Meara, the CEO of Tifco Hotels, which operates over 3,000 hotel beds. Additionally, there are opportunities to connect with Richard Collins, an Area Vice President of the Marriott Group, as well as contacts within several smaller hotel groups. These personal connections will help to reduce marketing expenses significantly during the initial phase and form strong partnerships with key customers.

These marketing strategies lead to the following sales forecast, as illustrated in Figure \ref{fig:sales}, with the units sold shown on the right-hand y-axis.


 \begin{figure}[ht]
     \centering
     \includegraphics[width=1\linewidth]{Pictures/Sales Forecast.png}
     \caption{Sales and Income Forecast}
     \label{fig:sales}
 \end{figure}

\subsection{Operations and Human Resources}


DrainPower aims to incorporate as a company under the UG (Unternehmergesellschaft) form in Germany by the end of the year. By early 2025, once the company’s value exceeds €50,000, it will automatically transition to a GmbH, the German equivalent of a limited liability company.

Regarding salaries, as outlined in Figure \ref{fig:opplan}, DrainPower’s operational strategy involves the founding team developing a patentable product by Q2 2025. Since all founding members will remain students for the upcoming year, they will not require a salary until early 2026. At that time, each founding member will earn an annual salary of €20,000 for the first two years. Once the product enters the pre-series phase, the team plans to hire a Mechanical Engineer at €60,000 per year and later a Sales Manager in Q3 2027. The team size will expand with the start of series production, and by Q1 2028, the founding team’s salaries will increase to €80,000 annually.

\begin{figure}[ht]
    \centering
    \includegraphics[width=1\linewidth]{Pictures/Operational Plan.png}
    \caption{HR Plan}
    \label{fig:opplan}
\end{figure}

\subsubsection{Strategic Business Development}
DrainPower’s initial strategy focuses on direct sales, targeting individual hotels before expanding to larger hotel chains. The goal is to establish cooperation with major hotel chains to integrate our solutions on a larger scale.

\subsubsection{Mission and Values}
Our mission is to prevent the thermal energy stored in drain water from going to waste. We are committed to developing sustainable solutions that contribute to energy efficiency and environmental preservation.

\subsubsection{Product and Service Development}
Our product offering includes the addition of a Heat Exchanger (HE) to existing Heat Pump (HP) systems. DrainPower will provide training for the installation and maintenance of these systems. Initially, production will be outsourced to external companies, with plans for vertical integration after series production to improve profit margins.


\subsubsection{Source and Production}
DrainPower will collaborate with an external company in Germany to manufacture the Heat Exchangers. Additionally, we will work with existing groundwater source heat pump manufacturers to purchase the required heat pumps.

\subsubsection{Storage and Logistics}
Initially, the Heat Exchangers can be stored in personal facilities (e.g., at home) until a dedicated storage facility is established. Delivery will also be outsourced to ensure timely and efficient distribution (Just in Time).

\subsubsection{Office and Infrastructure}
Office space will be arranged to support the operational needs of the company. This includes training facilities for installers and a design team responsible (Meie Kleijburg) for adapting the Heat Exchanger to various sites and ensuring compatibility with the purchased heat pumps, given the uniqueness of each installation scenario.

\subsubsection{Delivery and Installation}
The process involves the HP manufacturer delivering the groundwater heat pumps directly to the site. Simultaneously, DrainPower’s HE, along with a trained installer (external), will be deployed to ensure the seamless integration of the system.

\subsubsection{Customer Service and Continuous Improvement}
DrainPower will offer customer support through a chatbot and email, along with physical maintenance services on an annual basis. Additionally, a digital optimization platform will be developed to monitor and enhance system performance. The company plans to move towards vertical integration after the series production phase to increase profit margins and further improve operational efficiency.


\subsection{Financial Plan}
The financial plan for DrainPower outlines a strategic approach to investment, product development, and revenue generation over the next four years. The timeline for implementation and financial data provides a clear picture of how the company will grow, the capital required, and the expected revenue streams.

\begin{figure}[ht]
    \centering
    \includegraphics[width=1\linewidth]{Pictures/Financial Plan.png}
    \caption{Financial Plan}
    \label{fig:financial}
\end{figure}

\subsubsection{Revenue Projections}
DrainPower anticipates generating revenue primarily through the sale of drain water heat pumps, subscriptions to an optimisation platform, and service charges for maintenance. The revenue stream begins modestly in Q3 2025, with a total revenue of approximately €148,668. This is expected to grow significantly by 2027, reaching €2,911,798 in Q4 2027. The primary source of revenue will be the sale of drain water heat pumps, accounting for the majority of income, with substantial contributions also coming from the subscription service and maintenance charges.

\subsubsection{Cost of Goods and Services (COGS) and Gross Profit}
The Cost of Goods and Services (COGS) reflects the expenses associated with manufacturing the drain water heat pumps. Starting in Q3 2025, COGS are projected to be around €129,064, increasing with the scale of production. By 2027, COGS will peak at approximately €1,985,409. The gross profit margin is expected to improve over time, beginning at 13.2\% in 2025 and reaching 31.8\% by the end of 2027. This improvement in gross margin indicates that as production scales up, the company expects to achieve better economies of scale and operational efficiencies.

The initial pricing overview of the DWHP is shown in Figure \ref{fig:Pricing}. The discount value represents a reduced price for early products to compensate for more in-depth data collection and testing. Additionally, the maintenance charge will be set at €5000 annually, with €3000 to be paid to the installers, and the subscription to the maintenance platform will be set at €1000 annually.

\begin{figure}[ht]
    \centering
    \includegraphics[width=0.5\linewidth]{Pictures/Pricing.png}
    \caption{DWHP Pricing Breakdown}
    \label{fig:Pricing}
\end{figure}

\subsubsection{Operating Expenses}
Operating expenses will cover salaries and benefits, research and development (R\&D) investments, software hosting, general administrative costs, sales and marketing efforts, consulting fees, and participation in fairs and events. Notably, R\&D investments are front-loaded, with significant expenditures in 2025, including an outlay of €101,980 in Q1. Salaries and benefits are planned to commence in 2026, reflecting the hiring needed to support operations as the company moves towards commercial production. The total operating expenses are expected to rise steadily, peaking at €82,817 in Q4 2027.

\subsubsection{Financial Buffer and Working Capital Needs}
Given the nature of the business, particularly during the initial phase involving patenting and the commercial MVP (Minimum Viable Product), DrainPower will require a significant financial buffer as working capital. This buffer is essential for purchasing drain water heat pump units before they are sold to customers, representing the highest financial need, which exceeds €100,000.

To address this requirement, DrainPower plans to secure a 0\% interest loan of up to €250,000 from the Baden-Württemberg bank (BW Bank) under the Startfinanzierung 80 program \cite{l-bank_startfinanzierung_nodate}. This program is advantageous as it allows the state of Baden-Württemberg to cover 80\% of the loan guarantee, with the remaining 20\% possibly covered by Friends, Family, and Fools (FFF). Future loans, particularly as the company scales, could be taken up under the GuW-BW program, which offers interest-free loans for the first year, providing further financial flexibility \cite{l-bank_grundungs-_nodate}.

\subsubsection{Equity Investment Strategy}
DrainPower's equity investment strategy is designed to mitigate risk for investors and enhance company valuation. The company intends to seek equity investments after reaching critical milestones, such as filing patents and moving close to series production. This phased approach ensures that DrainPower demonstrates tangible progress before seeking additional capital, thereby increasing investor confidence and potentially securing better terms for equity financing.

\section{Economical and Financial Attractiveness for Investors}

DrainPower is an interesting investment opportunity, especially for those who want to support innovative and sustainable technologies. The company is in the renewable energy sector, which is growing because more people and companies are investigating how to be more energy-efficient and reduce carbon emissions.

\subsection{Strong Revenue Growth and Profit Margins}
DrainPower’s revenue is expected to grow steadily, starting from €148,668 in Q3 2025 and reaching nearly €2.91 million by Q4 2027. This growth will come mainly from selling drain water heat pumps, along with income from subscriptions to our optimization platform and maintenance services. Our gross profit margin is also expected to improve over time, starting at 13.2\% in 2025 and reaching 31.8\ by the end of 2027. This means that as we produce and sell more units, we will be able to make them more efficiently and increase our profits.

The combination of growing revenue and improving profit margins shows that DrainPower is on a good path towards financial sustainability and profitability. This is important for investors who are looking for a return on their investment in a relatively short period of time.

\subsection{Reduced Risk with Phased Investment and Milestones}
DrainPower is taking a careful approach to raising funds, planning to seek equity investment only after reaching key milestones, such as filing patents and getting closer to series production. This strategy reduces the risk for investors because it shows that the company is making real progress before asking for more capital. It also allows us to potentially increase our company valuation and get better terms when raising funds.

In the early stages, we are using non-dilutive funding sources, like an interest-free loan of up to €250,000 from the Baden-Württemberg bank (BW Bank) under the Startfinanzierung 80 program. This approach helps us keep financial flexibility and avoid diluting early investors’ shares too much.

\subsection{Financial Buffer and Working Capital}
DrainPower understands the importance of having a financial buffer, especially during the initial phases when we need to develop the product and enter the market. We plan to secure working capital to buy drain water heat pump units before selling them, which is a big financial need, more than €100,000. Having this buffer, along with our phased approach to investment, makes the company more stable and safer for investors.

\subsection{Market Potential and Competitive Advantage}
The market for energy-efficient heating solutions is growing fast because of new regulations and more awareness of environmental issues. DrainPower’s unique approach to capturing thermal energy from waste water gives us a strong competitive advantage in this market. Additionally, our use of AI-driven optimization makes our product more appealing because it offers not just a heating solution, but also a way to maximize energy efficiency.

\subsection{Estimated Valuation}
To estimate DrainPower’s valuation, we consider several factors, like projected revenue, market potential, and our current development stage. By the end of 2025, after we reach important milestones like patent filing and initial market entry, our valuation could be around €1.5 million to €3 million. This estimate considers the early-stage risks, our innovative product, and our growth potential.

As we move towards series production in 2027, with revenues close to €2.91 million and better profit margins, the valuation could increase significantly, potentially reaching between €7 million and €10 million. These estimates are based on common industry practices for early-stage tech companies, adjusted for the renewable energy sector \cite{arnaud_most_2018}.

\subsection{Conclusion}
DrainPower is an attractive investment opportunity, with strong growth prospects, a careful risk management approach, and a big market potential. Our phased investment strategy, combined with our innovative product and solid financial planning, makes DrainPower a promising choice for investors who want to be part of the future of energy efficiency.